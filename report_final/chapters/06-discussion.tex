\chapter{Discussion}

\section{Positioning Systems}
In terms of pure performance, the pressure-based system heavily outperforms every other proposed system, barring the Kalman filtering schemes tuned to prioritize pressure data. This is due to the high accuracy and sampling rate of the pressure data in our system, which when combined with the well-tuned altitude based mathematical spiral model, allows us to calculate accurate positions at a very high sampling rate. Due to this, the noisy data from the accelerometers and the Wi-Fi RFR weren't able to improve the predicted positions significantly. However, the constraints placed on the system in terms of the target environment are difficult to generalize to other use-cases, where a system relying only on pressure data wouldn't be able to determine the X and Y positions with a reasonable accuracy.
\par
Furthermore, if the spiral model used was mistuned or if the alignment was imperfect, the accuracy of the pressure-based system suffers significantly, which we encountered during our tuning. In cases where the environment model cannot deterministically predict an X, Y position for a given height, or in cases where the environment model is inaccurate or inconsistent, the Wi-Fi based Kalman filtering scheme can provide an additional layer of error correction and outperforms the pressure-only system. Furthermore, the entire pressure system heavily relies on the prior knowledge of the pressure at the starting position, which may not be known in some environments, or may vary. To deal with this, a separate module can be placed at a baseline position, in order to collect the pressure at the ground. An alternative method is to rely further on the Wi-Fi fingerprinting system, which does not rely on a predetermined pressure baseline value.

\section{Orientation Estimation Systems}
The orientation system proposed appears to perform well from a distance, with generally accurate orientation values along entire sequences. However, the system is prone to occasional drift issues, specifically on the heading axis. This is due to two main reasons. Firstly, the heading axis lacks an accurate baseline in the form of the gravity vector, which is only available on the other two axes. This issue can be mitigated via integrating a more accurate sensor or implementing a Kalman filtering/machine learning scheme to further improve the predicted angles. Secondly, the heading axis exhibits significantly larger variance than the other two axes of rotation, simply due to human anatomy and range of motion. However, the orientation estimation system generally performs well, with below 30 degrees of median angular error.