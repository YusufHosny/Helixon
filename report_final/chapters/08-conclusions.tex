\chapter{Conclusion}
In summary, the project successfully developed a robust real-time position and attitude estimation system tailored for the constrained indoor environment of KU Leuven's Group T Campus spiral walkway. The work implemented several techniques, such as altitude-based pressure models, Kalman filtering schemes, and machine learning approaches like Random Forest Regression, by fusing sensors with the Bosch BNO055 IMU, BMP390 pressure sensor, and Wi-Fi-based positioning. While the pressure-based system was the most accurate for vertical positioning, the Wi-Fi integration and Kalman filters did improve the accuracy in complex conditions. The attitude estimation, while reliable, had issues such as drift in the heading axis because of the inherent limitations in the IMU-based measurements.
 
Future work is directed at both hardware and algorithm improvements. For example, moving to more powerful microcontrollers, such as the ESP32, would support on-device processing and decrease latency and system complexity. Other improvements involve refining the generation of synthetic data, enhancing synchronization during preprocessing, and using higher-order positioning techniques, such as step counting and machine learning-driven spiral alignment. These improvements are aimed at solving current limitations, enhancing generalizability, and ensuring the robustness of the system in various real-world applications.
