\chapter{Related Work}
The task of IMU-based head-tracking and positioning is a decades old task with significant literature exploring many different methods and solutions. A large portion of existing literature exploits knowledge of the system's behavioral model and multiple measurement sensors to implement Kalman filtering schemes \cite{sabatini_2011_kalmanfilterbased, hellmers_2013_an}. More recent methods also include step detection and counting methods in order to reduce drift issues in dead reckoning applications \cite{tiwari_2022_a, huang_2022_improvement}. Finally, with the modern advancements in compute and machine learning algorithms, machine learming methods have been implemented to create end-to-end solutions for position and orientation tracking using machine learning models and large labelled datasets. Additionally, using IMU data isn't the only proposed solution for cost-effective position tracking systems, with many system exploiting the ubiquity of WLAN/Wi-Fi to incorporate signal strength information into their system for additional positioning accuracy. \cite{stanculeanu_2012_enhanced, shang_2014_a}.